%!TEX TS-program = xelatex
%!TEX encoding = UTF-8 Unicode

% Author:       Artiom Basenko
% Version:      0.1
% Fonts:        Libertine G, O; Biolinum O, Arno Pro
% Languages:    Русский, English
% Contents:     Curriculum vitae (на русском)

%%%%%%%%%%%%%
%  Options  %
%%%%%%%%%%%%%

% Basics %

% Paper options using the scrartcl class
\documentclass[a4paper, oneside, final]{scrartcl}

% Tweak document geometry a little
\usepackage[body={6.8in, 9.3in}, top=1.1in]{geometry}

% Multilanguage support (fontspec included)
\usepackage{polyglossia}
\setmainlanguage{russian}
\setotherlanguage{english}

\usepackage{scrpage2} % Provides headers and footers configuration
\usepackage{titlesec} % Allows creating custom \section's
\usepackage{marvosym} % Allows the use of symbols
\usepackage{tabularx,colortbl} % Advanced table configurations
\defaultfontfeatures{Mapping=tex-text} % Use Tex mappings
\defaultfontfeatures{Ligatures=TeX} % Use Tex dashes

% Pretty fonts
% \newfontfamily\russianfont[Script=Cyrillic]{Linux Libertine O}
% \newfontfamily\englishfont[Script=Latin]{Linux Biolinum O}

% Alternatives
\newfontfamily\russianfont[Script=Cyrillic]{Gentium Plus}

% Advanced %

% Section formatting
\titleformat{\section}{\large\scshape\raggedright}{}{0em}{}[\titlerule]
\pagestyle{scrheadings} % Print the headers and footers on all pages

% Set minimal list item spacing
\usepackage{enumitem}
\setlist{nolistsep}

% Adjust the vertical offset - less whitespace at the top of the page
\addtolength{\voffset}{-0.7in}
% Adjust the text height - less whitespace at the bottom of the page
\addtolength{\textheight}{3cm}

% Custom highlighting for the work experience and education sections
\newcommand{\gray}{\rowcolor[gray]{.90}}

% Allow additional lines on the page
% \enlargethispage{-5\baselineskip}

%%%%%%%%%%%%
%  Footer  %
%%%%%%%%%%%%

% Font settings for footer
\renewcommand{\headfont}{\normalfont\rmfamily\scshape}

\cofoot{
    % Letter spacing and font size
    \addfontfeature{LetterSpace=10.0}\fontsize{12.5}{17}\selectfont
    % Email address and phone number
    {\Large\Letter} demi.log@gmail.com \ {\Large\Telefon} +7(904) 551-1140
}

% Document START %

\begin{document}

\begin{center} % Center everything!

%%%%%%%%%%%%
%  Header  %
%%%%%%%%%%%%

{\addfontfeature{LetterSpace=10.0}\fontsize{26}{26}\selectfont\scshape
    Басенко Артем Сергеевич
}

\vspace{0.5cm} % Extra whitespace after title

%%%%%%%%%%%%
% Personal %
%%%%%%%%%%%%

\section{Личные данные}

\begin{tabular}{ @{} >{\bfseries}l @{\hspace{6ex}} l }
    Год рождения & 1988 \\
    Место проживания & Сертолово, ул. Кленовая (ст. м. "Проспект Просвещения")\\
    Телефон & +7(904)551-1140 \\
    Email & demi.log@gmail.com
\end{tabular}

%%%%%%%%%%%%%
% Objective %
%%%%%%%%%%%%%

\section{Цель}

Получение должности разработчика веб-приложений. \\
Интересует разработка как серверной так и пользовательской части.

%%%%%%%%%%%%%%
% Experience %
%%%%%%%%%%%%%%

\section{Опыт работы}

\begin{tabularx}{0.97\linewidth}{>{\raggedleft\scshape}p{2.7cm}X}
    \gray Период & \textbf{Ноябрь 2013 -- настоящее время}\\
    \gray Работодатель & \textbf{Фриланс}\\
    % \gray Должность & \textbf{Веб-программист}\\
    \gray Технологии & \textbf{PHP, AngularJS, HTML, SASS}\\
    % \hline
    \center &
        \begin{itemize}
            \item Разработка фронтенда и бэкенда для новостного ресурса
            \item Создание пользовательской части "виртуального репетитора"
            \item Профилирование и оптимизация приложения
            \item Настройка сервера и системы непрерывной интеграции
        \end{itemize}
\end{tabularx}

\vspace{6pt}

\begin{tabularx}{0.97\linewidth}{>{\raggedleft\scshape}p{2.7cm}X}
    \gray Период & \textbf{Август 2011 -- ноябрь 2013}\\
    \gray Работодатель & \textbf{ООО Управление Механизации}\\
    % \gray Должность & \textbf{Веб-программист}\\
    \gray Технологии & \textbf{PHP, JavaScript, HTML, CSS, ActionScript, Java, Android}\\
    % \hline
    \center &
        \begin{itemize}
            \item Проектирование и реализация архитектуры распределённой системы документооборота
            \item Разработка веб-сервиса электронных подарков на основе QR-кодов
            \item Реализация API для мобильных и Flex приложений
            \item Создание мобильного приложения и встраиваемого Flash-модуля
            \item Настройка системы непрерывной интеграции
            \item Создание задач для сервера Gearman
        \end{itemize}
\end{tabularx}

%%%%%%%%%%%%%
% Education %
%%%%%%%%%%%%%

\section{Образование}

\begin{tabularx}{0.97\linewidth}{>{\raggedleft\scshape}p{2.7cm}X}
    \gray Период & \textbf{2005 -- 2011} \\
    \gray Степень & \textbf{Магистр информационных технологий} \\
    \gray Университет & \textbf{Государственный политехнический (СПбГПУ)} \\
    \gray Факультет & \textbf{Технической Кибернетики (ФТК)} \\
    \gray Специальность & \textbf{Инженер-проектировщик дискретных устройств}
\end{tabularx}

%%%%%%%%%%%%%%
%   Skills   %
%%%%%%%%%%%%%%

\section{Навыки}

\begin{tabular}{ @{} >{\bfseries}l @{\hspace{6ex}} l }
    Операционные системы & Windows, Linux (Debian, Arch, CentOS), OS X \\
    Языки программирования & PHP, Python, JavaScript, Java \\
    Протоколы, форматы и API & XML, JSON, YAML, SOAP, REST \\
    Базы данных & MySQL, PostgreSQL, Redis, MongoDB \\
    Инструментарий & Git, Mercurial, SVN, Vim, Eclipse, IntelliJ, Vagrant, LaTeX \\
    Библиотеки & ZendFramework, Django, Bottle, AngularJS, SASS, Grunt, Selenium \\
    Иностранные языки & Английский (advanced), Японский (JLPT N2), Испанский (DELE A1)
\end{tabular}

% Document END %

\end{center}

\end{document}
